\documentclass{article}

\usepackage{../../../../latex_styles/standard-3}
\usepackage{tikz}
\def\ds{\displaystyle}

\setlength{\parindent}{0.0pt}

\begin{document}
%The following assumes a piecewise linear test function, $u\in[-1,1]$, defined over a reference element:
%\begin{figure}[h!]
%\begin{center}
% \resizebox{2.5in}{!}{
%  \begin{tikzpicture}
%    \draw[color=lightgray, very thin] (-.99,-.05) grid (0.99,0.99);
%    \foreach \x in {-1,...,1} {
%        \draw [very thin, black] (\x, 0.0) -- (\x , 0.0) node [below] at (\x,0.0) {\tiny{$\x$}};
%        }
%    \foreach \y in {1} {
%        \draw [very thin, black] (0,\y) -- (0,\y) node [right] at (0,\y) {\tiny{$\y$}};
%        }
%    \draw[ thick, blue,domain=-1.:1.] plot(\x,{-1/2*(\x-1)})  node[] at (-1.25,1.1) {\tiny{$\lambda_1$}};
%    \draw[ thick, red,domain=-1.:1.] plot(\x,{1/2*(\x+1)})  node[] at (1.25,1.1)  {\tiny{$\lambda_2$}};
%  \end{tikzpicture}}
%  \caption{Piecewise linear basis functions. $\lambda_1 = -\frac{1}{2}(\hat{x}-1)$ and $\lambda_2 = \frac{1}{2}(\hat{x}+1)$}
%  \label{fig::PiecewiseLinearBasis}
%\end{center}
%\end{figure}
%
%Example 1D mesh:
%\begin{figure}[h!]
%\begin{center}
% \resizebox{2.5in}{!}{
%  \begin{tikzpicture}
%    \draw[color=lightgray, very thin] (-1.0,-.05) grid (1.,0.99);
%%    \foreach \x in {-1,...,1} {
%%        \draw [very thin, black] (\x, 0.0) -- (\x , 0.0) node [below] at (\x,0.0) {\tiny{$\x$}};
%%        }
%%    \foreach \y in {1} {
%%        \draw [very thin, black] (0,\y) -- (0,\y) node [right] at (0,\y) {\tiny{$\y$}};
%%        }
%    \draw[ thick, blue,domain=-1.:0.] plot(\x,{(\x+1)})  node[] at (-.20,1.0) {\tiny{$\mu_1$}};
%    \draw[ thick, red,domain=-1.:0] plot(\x,{(-\x)})  node[] at (-.75,1.0)  {\tiny{u$_2$}};
%%   \draw[ thick, blue,domain=0.:1.0] plot(\x,{(\x)})  node[] at (-.75,1.1) {\tiny{u$_1$}};
%%    \draw[ thick, red,domain=0:1.0] plot(\x,{(-\x+1)})  node[] at (.75,1.1)  {\tiny{u$_2$}};
%  \end{tikzpicture}}
%  \caption{Piecewise linear basis functions. $\lambda_1 = -\frac{1}{2}(\hat{x}-1)$ and $\lambda_2 = \frac{1}{2}(\hat{x}+1)$}
%  \label{fig::PiecewiseLinearBasis}
%\end{center}
%\end{figure}

Consider the 1D wave equation:
\begin{equation}
\frac{dT}{dt} - \lambda T = S \\[5pt]
\end{equation}

Multiply by a test function, $u$, and integrate over phase space, $D^k$:
\begin{equation}
\ds\int_{\Omega} u \frac{dT}{dt} dD^k - \ds\int_{\Omega}  \lambda u T dD^k = \ds\int_{\Omega} u S dD^k\\[5pt]
\end{equation}

Integrating the first term by parts:
\begin{gather}
\left( uT\big|_{j-1/2}^{j+1/2} - \ds\int_{\Omega} T \frac{du}{dt}dD^k \right) - \ds\int_{\Omega} \lambda u TdD^k =  \ds\int_{\Omega} u SdD^k \\[5pt]
uT_{j+1/2} - uT_{j-1/2} - \ds\int_{\Omega} T \frac{du}{dt} dD^k- \ds\int_{\Omega} \lambda u T dD^k= \ds\int_{\Omega} uSdD^k
\end{gather}

Where $j-1/2$ and $j+1/2$ are the surfaces of cell, $j$. We can assume a separable solution for $T=\ds\sum_{q=0}^R \bm{\tau} u_q$, with $\bm{\tau} = \tau_0,\dots,\tau_R$
\begin{equation}
\label{eq::five}
\ds\sum_{q=0}^R \left[ u_qT_{j+1/2} - u_qT_{j-1/2} - \ds\int_{\Omega}u_{j}'u_q \bm{\tau} - \ds\int_{\Omega} \lambda u_j u_q \bm{\tau}\right] = \ds\int_{\Omega} u_jS
\end{equation}

At this point it is instructive to focus on each of the terms in the above equation individually. If we expand the surface evaluations in terms of the individual basis functions, $u_0,\dots,u_r$ (using piecewise polynomials, $r=1$), the following can be obtained:
\begin{equation}
\label{eq::Surf}
\begin{matrix}
u_0T_{j+1/2} - u_0T_{j-1/2} \\[5pt]
u_1T_{j+1/2} - u_1T_{j-1/2} 
\end{matrix} 
\quad \to \quad 
\begin{bmatrix} -1 & 0 \\ 0 & 1 \end{bmatrix} \begin{bmatrix} T_{j-1/2}  \\ T_{j+1/2}  \end{bmatrix}
\end{equation}

Now shifting attention to the volume integrals:
\begin{align}
\label{eq::Stiff}
\ds\int_\Omega u_{j}'u_q \bm\tau \quad &\to \quad \begin{bmatrix} -1/2 & -1/2 \\ 1/2 & 1/2  \end{bmatrix} \begin{bmatrix} \tau_0 \\ \tau_1 \end{bmatrix} \\[5pt]
\label{eq::Mass}
\lambda\ds\int_\Omega u_{j}u_q \bm\tau \quad &\to \quad \lambda\begin{bmatrix} 2/3 & 1/3 \\ 1/3 & 2/3  \end{bmatrix} \begin{bmatrix} \tau_0 \\ \tau_1 \end{bmatrix}
\end{align}

Substituting Equations \ref{eq::Surf}, \ref{eq::Stiff}, and \ref{eq::Mass} into Equation \ref{eq::five}:
\begin{equation}
\label{eq::Almost}
\begin{bmatrix} -1 & 0 \\ 0 & 1 \end{bmatrix} \begin{bmatrix} T_{j-1/2}  \\ T_{j+1/2}  \end{bmatrix} - \begin{bmatrix} -1/2 & -1/2 \\ 1/2 & 1/2  \end{bmatrix} \begin{bmatrix} \tau_0 \\ \tau_1 \end{bmatrix} - \lambda\begin{bmatrix} 2/3 & 1/3 \\ 1/3 & 2/3  \end{bmatrix} \begin{bmatrix} \tau_0 \\ \tau_1 \end{bmatrix} = \begin{bmatrix} S_0 \\ S_1 \end{bmatrix}
\end{equation}

However, Equation \ref{eq::Almost} has a total of 4 unknowns (the surface terms $T_{j+1/2}$ and $T_{j-1/2}$, and the coefficients $\tau_0$ and $\tau_1$) and only two equations. In order to overcome this, we will make the following assumptions. By using upwinding we can set $T_{j+1/2}$ to be $\tau_1$ and $T_{j-1/2}$ to be $\tau_1^{k-1}$ where the superscript $k-1$ denotes the neighboring (upwind) element. Note, that $\tau_1^{k-1}$ is a \textit{known} value and can be shifted to the right hand side; for the first element, $\tau_1^{k-1}$, is the initial condition.
\begin{equation}
\begin{bmatrix} 0 & 0 \\ 0 & 1 \end{bmatrix} \begin{bmatrix} \tau_0 \\ \tau_1 \end{bmatrix} - \begin{bmatrix} -1/2 & -1/2 \\ 1/2 & 1/2  \end{bmatrix} \begin{bmatrix} \tau_0 \\ \tau_1 \end{bmatrix} - \lambda\begin{bmatrix} 2/3 & 1/3 \\ 1/3 & 2/3  \end{bmatrix} \begin{bmatrix} \tau_0 \\ \tau_1 \end{bmatrix} = \begin{bmatrix} S_0+\tau_1^{k-1} \\ S_1 \end{bmatrix}
\end{equation}
Now, collecting terms we end with a $2\times 2$ ($R\times R$ for general polynomials) system of equations to solve for each element:
\begin{equation}
\frac{1}{6}\begin{bmatrix} 3-4\lambda & 3-2\lambda \\ -3-2\lambda & 3-4\lambda \end{bmatrix} \begin{bmatrix} \tau_0 \\ \tau_1 \end{bmatrix} =  \begin{bmatrix} S_0+\tau_1^{k-1} \\ S_1 \end{bmatrix}
\end{equation}



\end{document}